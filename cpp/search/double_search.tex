双向同时搜索:
\begin{lstlisting}
将开始结点和目标结点加入队列 q
标记开始结点为 1
标记目标结点为 2
while (队列 q 不为空)
{
	从 q.front() 扩展出新的 s 个结点
	
	如果 新扩展出的结点已经被其他数字标记过
	那么 表示搜索的两端碰撞
	那么 循环结束
	
	如果 新的 s 个结点是从开始结点扩展来的
	那么 将这个 s 个结点标记为 1 并且入队 q 
	
	如果 新的 s 个结点是从目标结点扩展来的
	那么 将这个 s 个结点标记为 2 并且入队 q
}
\end{lstlisting}
Meet in the middle

Meet in the middle 算法的主要思想是将整个搜索过程分成两半,分别搜索,最后将两半的结果合并。

它适用于输入数据较小,但还没小到能直接使用暴力搜索的情况。

暴力搜索的复杂度往往是指数级的,而改用 meet in the middle 算法后复杂度的指数可以减半,即让复杂度从 $O(a^b)$ 降到 $O(a^{b/2})$。

例题:

有 $n$ 盏灯,每盏灯与若干盏灯相连,每盏灯上都有一个开关,如果按下一盏灯上的开关,这盏灯以及与之相连的所有灯的开关状态都会改变。一开始所有灯都是关着的,你需要将所有灯打开,求最小的按开关次数。

$1\le n\le 35$。
\begin{lstlisting}
#include <algorithm>
#include <cstdio>
#include <iostream>
#include <map>

using namespace std;

int n, m, ans = 0x7fffffff;
map<long long, int> f;
long long a[40];

int main() {
	cin >> n >> m;
	a[0] = 1;
	for (int i = 1; i < n; ++i) a[i] = a[i - 1] * 2;  // 进行预处理
	
	for (int i = 1; i <= m; ++i) {  // 对输入的边的情况进行处理
		int u, v;
		cin >> u >> v;
		--u;
		--v;
		a[u] |= ((long long)1 << v);
		a[v] |= ((long long)1 << u);
	}
	
	for (int i = 0; i < (1 << (n / 2)); ++i) {  // 对前一半进行搜索
		long long t = 0;
		int cnt = 0;
		for (int j = 0; j < n / 2; ++j) {
			if ((i >> j) & 1) {
				t ^= a[j];
				++cnt;
			}
		}
		if (!f.count(t))
		f[t] = cnt;
		else
		f[t] = min(f[t], cnt);
	}
	
	for (int i = 0; i < (1 << (n - n / 2)); ++i) {  // 对后一半进行搜索
		long long t = 0;
		int cnt = 0;
		for (int j = 0; j < (n - n / 2); ++j) {
			if ((i >> j) & 1) {
				t ^= a[n / 2 + j];
				++cnt;
			}
		}
		if (f.count((((long long)1 << n) - 1) ^ t))
		ans = min(ans, cnt + f[(((long long)1 << n) - 1) ^ t]);
	}
	
	cout << ans;
	
	return 0;
}
\end{lstlisting}