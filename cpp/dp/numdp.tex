\begin{lstlisting}
typedef long long ll;
int a[20];
ll dp[20][state];//不同题目状态不同
ll dfs(int pos,/*state变量*/,bool lead/*前导零*/,bool limit/*数位上界变量*/)//不是每个题都要判断前导零
{
	//递归边界,既然是按位枚举,最低位是0,那么pos==-1说明这个数我枚举完了
	if(pos==-1) return 1;/*这里一般返回1,表示你枚举的这个数是合法的,那么这里就需要你在枚举时必须每一位都要满足题目条件,也就是说当前枚举到pos位,一定要保证前面已经枚举的数位是合法的。不过具体题目不同或者写法不同的话不一定要返回1 */
	//第二个就是记忆化(在此前可能不同题目还能有一些剪枝)
	if(!limit && !lead && dp[pos][state]!=-1) return dp[pos][state];
	/*常规写法都是在没有限制的条件记忆化,这里与下面记录状态是对应,具体为什么是有条件的记忆化后面会讲*/
	int up=limit?a[pos]:9;//根据limit判断枚举的上界up;这个的例子前面用213讲过了
	ll ans=0;
	//开始计数
	for(int i=0;i<=up;i++)//枚举,然后把不同情况的个数加到ans就可以了
	{
		if() ...
		else if()...
		ans+=dfs(pos-1,/*状态转移*/,lead && i==0,limit && i==a[pos]) //最后两个变量传参都是这样写的
		/*这里还算比较灵活,不过做几个题就觉得这里也是套路了
		大概就是说,我当前数位枚举的数是i,然后根据题目的约束条件分类讨论
		去计算不同情况下的个数,还有要根据state变量来保证i的合法性,比如题目
		要求数位上不能有62连续出现,那么就是state就是要保存前一位pre,然后分类,
		前一位如果是6那么这意味就不能是2,这里一定要保存枚举的这个数是合法*/
	}
	//计算完,记录状态
	if(!limit && !lead) dp[pos][state]=ans;
	/*这里对应上面的记忆化,在一定条件下时记录,保证一致性,当然如果约束条件不需要考虑lead,这里就是lead就完全不用考虑了*/
	return ans;
}

ll solve(ll x)
{
	int pos=0;
	while(x)//把数位都分解出来
	{
		a[pos++]=x%10;//个人老是喜欢编号为[0,pos),看不惯的就按自己习惯来,反正注意数位边界就行
		x/=10;
	}
	return dfs(pos-1/*从最高位开始枚举*/,/*一系列状态 */,true,true);//刚开始最高位都是有限制并且有前导零的,显然比最高位还要高的一位视为0嘛
}

int main()
{
	ll le,ri;
	while(~scanf("%lld%lld",&le,&ri))
	{
		//初始化dp数组为-1,这里还有更加优美的优化,后面讲
		printf("%lld\n",solve(ri)-solve(le-1));
	}
}
\end{lstlisting}