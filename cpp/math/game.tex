\begin{enumerate}
	\item Nim Game
	最经典最基础的博弈.
	n堆石子,双方轮流从任意一堆石子中取出至少一个,不能取的人输.
	对于一堆x个石子的情况,容易用归纳法得到SG(x)=x.
	所以所有石子个数的异或和为0是必败态,否则为必胜态.
	
	\item Bash Game
	每人最多一次只能取m个石子,其他规则同Nim Game.
	依旧数学归纳…SG(x)=xmod(m+1).
	
	\item NimK Game
	每人一次可以从最多K堆石子中取出任意多个,其他规则同Nim Game.
	结论:在二进制下各位上各堆石子的数字之和均为(K+1)的倍数的话则为必败态,否则为必胜态.
	这个证明要回到原始的方法上去.
	补:这个游戏还可以推广,即一个由n个子游戏组成的游戏,每次可以在最多K个子游戏中进行操作.
	然后只要把结论中各堆石子的个数改为各个子游戏的SG值即可,证明也还是一样的.
	
	\item Anti-Nim Game
	似乎又叫做Misère Nim.
	不能取的一方获胜,其他规则同Nim Game.
	关于所谓的”Anti-SG游戏”及”SJ定理”贾志鹏的论文上有详细说明,不过似乎遇到并不多.
	结论是一个状态是必胜态当且仅当满足以下条件之一:
	SG值不为0且至少有一堆石子数大于1;
	SG值为0且不存在石子数大于1的石子堆.
	
	\item Fibonacci Nim
	有一堆个数为n(n>=2)的石子,游戏双方轮流取石子,规则如下:
	1)先手不能在第一次把所有的石子取完,至少取1颗;
	2)之后每次可以取的石子数至少为1,至多为对手刚取的石子数的2倍。
	约定取走最后一个石子的人为赢家。
	结论:当n为Fibonacci数的时候,必败。
	
	\item Staircase Nim
	每人一次可以从第一堆石子中取走若干个,或者从其他石子堆的一堆中取出若干个放到左边一堆里(没有石子的石子堆不会消失),其他规则同Nim Game.
	这个游戏的结论比较神奇:
	当且仅当奇数编号堆的石子数异或和为0时为必败态.
	简单的理解是从偶数编号堆中取石子对手又可以放回到奇数编号堆中,而且不会让对手不能移动.比较意识流,然而可以归纳证明.
	
	\item Wythoff Game
	有两堆石子,双方轮流从某一堆取走若干石子或者从两堆中取走相同数目的石子,不能取的人输.
	容易推理得出对任意自然数k,都存在唯一的一个必败态使得两堆石子数差为k,设其为Pk=(ak,bk),表示石子数分别为ak,bk(ak<=bk).
	那么ak为在Pk0(k0<k)中未出现过的最小自然数,bk=ak+k.
	数学班的说,用Betty定理以及显然的单调性就可以推出神奇的结论:
	ak=floor(k*5√+12),bk=floor(k*5√+32).
	
	\item Take \& Break
	有n堆石子,双方轮流取出一堆石子,然后新增两堆规模更小的石子堆(可以没有石子),无法操作者输.
	这个游戏似乎只能暴力SG,知道一下就好.
	
	\item 树上删边游戏
	给出一个有n个结点的树,有一个点作为树的根节点,双方轮流从树中删去一条边边,之后不与根节点相连的部分将被移走,无法操作者输.
	结论是叶子结点的SG值为0,其他结点SG值为其每个儿子结点SG值加1后的异或和,证明也并不复杂.
	
	\item 翻硬币游戏
	n枚硬币排成一排,有的正面朝上,有的反面朝上。
	游戏者根据某些约束翻硬币(如:每次只能翻一或两枚,或者每次只能翻连续的几枚),但他所翻动的硬币中,最右边的必须是从正面翻到反面。
	谁不能翻谁输。
	
	\item 需要先开动脑筋把游戏转化为其他的取石子游戏之类的,然后用如下定理解决:
	局面的 SG 值等于局面中每个正面朝上的棋子单一存在时的 SG 值的异或和。
	
	\item 无向图删边游戏
	一个无向连通图,有一个点作为图的根。
	游戏者轮流从图中删去边, 删去一条边后,不与根节点相连的部分将被移走。
	谁无路可走谁输。
	
	\item 对于这个模型,有一个著名的定理——Fusion Principle:
	我们可以对无向图做如下改动:将图中的任意一个偶环缩成一个新点,任意一个奇环缩成一个新点加一个新边;所有连到原先环上的边全部改为与新点相连。 这样的改动不会影响图的 SG 值。
\end{enumerate}

SG函数:
\begin{lstlisting}
/* 
* author:Simon
* f[m]:可改变当前状态的方式,m为方式的种类,f[m]要在getSG之前先预处理
* sg[]:0~n的SG函数值
* mex[]:为x后继状态的集合
* 若sg值为正数,则先手必赢,否则若为0,则先手必输。
*/
int f[maxn],sg[maxn],mex[maxn];
void getSG(int n/*需要求多少个sg值*/,int m/*有多少种操作方式*/){
	memset(sg,0,sizeof(sg));
	for(int i = 1; i <= n; i++){ /*因为SG[0]始终等于0,所以i从1开始*/
		memset(mex,0,sizeof(mex)); /*每一次都要将上一状态 的 后继集合 重置*/
		for(int j = 0; f[j] <= i && j < m; j++)
		mex[sg[i-f[j]]] = 1;  /*将后继状态的SG函数值进行标记*/
		for(int j = 0;; j++) if(!mex[j]){   /*查询当前后继状态SG值中最小的非零值*/
			sg[i] = j;
			break;
		}
	}
}
\end{lstlisting}