一天,小徐的好友邀请他去吃布丁,于是小徐高高兴兴的来到好友家。

哇,这么多五彩缤纷的布丁!

好友说:“在我们开吃前先玩会儿游戏吧。”

于是他将布丁摆成一行,接着说:“我可以把某种颜色的布丁全部变成另一种颜色,我还会在某些时刻问你当前一共有多少段颜色。例如:颜色分别为 1,2,2,1 的四个布丁一共有 3 段颜色。”

输入格式
第一行包含整数 n 和 m,分别表示布丁的个数和好友的操作次数。

第二行包含 n 个空格隔开的整数 A1,A2,…,An,其中 Ai 表示第 i 个布丁的颜色。

从第三行起的 m 行,依次描述 m 个操作。

对每个操作,若第一个数是 1,则表示好友要改变颜色,这时后跟两个整数 x 和 y(可能相等),表示执行该操作后所有颜色为 x 的布丁被变成颜色 y。

若第一个数是 2,则表示好友要询问目前有多少段颜色,这时应该输出一个整数回答。

输出格式
对于每个询问,在一行中输出一个整数作为回答。

数据范围
0<n,m<100001,
0<Ai,x,y<106

\begin{lstlisting}
#include<bits/stdc++.h>

using namespace std;

#define int long long

const int N = 1e5 + 10;

int n, m;
int rt, idx;

struct type {
	int s[2], v, p;
	int sz, flag;
	int& l = s[0], & r = s[1];
	void init(int vv, int pp) {
		v = vv, p = pp;
		sz = 1;
	}
}tree[N];

void pushup(int x) {
	tree[x].sz = tree[tree[x].l].sz + tree[tree[x].r].sz + 1;
}

void pushdown(int x) {
	if (tree[x].flag) {
		swap(tree[x].l, tree[x].r);
		tree[tree[x].l].flag ^= 1;
		tree[tree[x].r].flag ^= 1;
		tree[x].flag = 0;
	}
}

void rotate(int x) {
	int y = tree[x].p, z = tree[y].p, k = tree[y].r == x;
	int& b = tree[x].s[k ^ 1];
	tree[b].p = y; tree[y].p = x; tree[x].p = z;
	tree[y].s[k] = b;
	b = y;
	tree[z].s[tree[z].r == y] = x;
	pushup(y); pushup(x);
}

void splay(int x, int k) {
	while (tree[x].p != k) {
		int y = tree[x].p, z = tree[y].p;
		if (z != k) {
			if ((tree[y].r == x) ^ tree[z].r == y) rotate(x);
			else rotate(y);
		}
		rotate(x);
	}
	if (!k) rt = x;
}

void insert(int v) {
	int x = rt, p = 0;
	while (x) p = x, x = tree[x].s[v > tree[x].v];
	x = ++idx;
	if (p) tree[p].s[v > tree[p].v] = x;
	tree[x].init(v, p);
	splay(x, 0);
}

int getp(int k) {
	int x = rt;
	while (1) {
		pushdown(x);
		int sz = tree[tree[x].l].sz;
		if (sz >= k) x = tree[x].l;
		else if (sz + 1 == k) return x;
		else k -= sz + 1, x = tree[x].r;
	}
}

void print(int x) {
	pushdown(x);
	if (tree[x].l) print(tree[x].l);
	if (tree[x].v >= 1 && tree[x].v <= n) cout << tree[x].v << " ";
	if (tree[x].r) print(tree[x].r);
} 

void oper() {
	cin >> n >> m;
	for (int i = 0; i <= n + 1; i++) insert(i);
	for (int i = 1; i <= m; i++) {
		int l, r; cin >> l >> r;
		l = getp(l), r = getp(r + 2);
		splay(l, 0), splay(r, l);
		tree[tree[r].l].flag ^= 1;
	}
	print(rt);
}

signed main() {
	ios::sync_with_stdio(0), cin.tie(0), cout.tie(0);
	int t = 1; //cin >> t;
	while (t--) oper();
	return 0;
}
\end{lstlisting}