普通线段数。单调修改、查询区间和:
\begin{lstlisting}
template <class Info, class Merge = std::plus<Info>> 
struct SegmentTree {
	const int n;
	const Merge merge;
	std::vector<Info> info;
	
	SegmentTree(int n) : 
	n(n), merge(Merge()), info(4 << (31 - __builtin_clz(n))) {}
	SegmentTree(std::vector<Info> init) : 
	SegmentTree(init.size()) {
		std::function<void(int, int, int)> build = [&](int p, int l, int r) {
			if (r - l == 1) {
				info[p] = init[l];
				return;
			}
			int m = (l + r) / 2;
			build(2 * p, l, m);
			build(2 * p + 1, m, r);
			pull(p);
		};
		build(1, 0, n);
	}
	void pull(int p) { info[p] = merge(info[2 * p], info[2 * p + 1]); }
	void modify(int p, int l, int r, int x, const Info &v) {
		if (r - l == 1) {
			info[p] = info[p] + v;
			return;
		}
		int m = (l + r) / 2;
		if (x < m)
			modify(2 * p, l, m, x, v);
		else
			modify(2 * p + 1, m, r, x, v);
		pull(p);
	}
	void modify(int p, const Info &v) { modify(1, 0, n, p, v); }
	Info rangeQuery(int p, int l, int r, int x, int y) {
		if (l >= y || r <= x) 
			return Info();
		if (l >= x && r <= y) 
			return info[p];
		int m = (l + r) / 2;
		return 
			merge(rangeQuery(2 * p, l, m, x, y), rangeQuery(2 * p + 1, m, r, x, y));
	}
	Info rangeQuery(int l, int r) { 
		return rangeQuery(1, 0, n, l, r); 
	}
};

struct Info {
	int64_t x;
	Info(int x = 0) : x(x){};
};

Info operator+(const Info &lhs, const Info &rhs) { 
	return lhs.x + rhs.x;
}
\end{lstlisting}
懒标记。区间修改和查询:
\begin{lstlisting}
#include <bits/stdc++.h>

struct Info {
	int64_t x;
	Info(int64_t x = 0) : x(x) {}
};

Info operator+(const Info &lhs, const Info &rhs) { return Info(lhs.x + rhs.x); }

void apply(Info &a, int64_t b) { a.x += b; }
void apply(int64_t &a, int64_t b) { a += b; }

template <class Info, class Tag, class Merge = std::plus<Info>> struct LazySegmentTree {
	const int n;
	const Merge merge;
	std::vector<Info> info;
	std::vector<Tag> tag;
	
	LazySegmentTree(int n) : n(n), merge(Merge()), info(4 << (31 - __builtin_clz(n))), tag(4 << (31 - __builtin_clz(n))) {}
	LazySegmentTree(std::vector<Info> init) : LazySegmentTree(init.size()) {
		std::function<void(int, int, int)> build = [&](int p, int l, int r) {
			if (r - l == 1) {
				info[p] = init[l];
				return;
			}
			int m = (l + r) / 2;
			build(2 * p, l, m);
			build(2 * p + 1, m, r);
			pull(p);
		};
		build(1, 0, n);
	}
	void pull(int p) { info[p] = merge(info[2 * p], info[2 * p + 1]); }
	void apply(int p, int l, int r, const Tag &v) {
		::apply(info[p], (r - l) * v);
		::apply(tag[p], v);
	}
	void push(int p, int l, int r) {
		int mid = (l + r) / 2;
		apply(2 * p, l, mid, tag[p]);
		apply(2 * p + 1, mid, r, tag[p]);
		tag[p] = Tag();
	}
	void modify(int p, int l, int r, int x, const Info &v) {
		if (r - l == 1) {
			info[p] = info[p] + v;
			return;
		}
		int m = (l + r) / 2;
		push(p, l, r);
		if (x < m)
		modify(2 * p, l, m, x, v);
		else
		modify(2 * p + 1, m, r, x, v);
		pull(p);
	}
	void modify(int p, const Info &v) { modify(1, 0, n, p, v); }
	Info rangeQuery(int p, int l, int r, int x, int y) {
		if (l >= y || r <= x) {
			return Info();
		}
		if (l >= x && r <= y) {
			return info[p];
		}
		int m = (l + r) / 2;
		push(p, l, r);
		return merge(rangeQuery(2 * p, l, m, x, y), rangeQuery(2 * p + 1, m, r, x, y));
	}
	Info rangeQuery(int l, int r) { return rangeQuery(1, 0, n, l, r); }
	void rangeApply(int p, int l, int r, int x, int y, const Tag &v) {
		if (l >= y || r <= x) return;
		if (l >= x && r <= y) {
			apply(p, l, r, v);
			return;
		}
		int m = (l + r) / 2;
		push(p, l, r);
		rangeApply(2 * p, l, m, x, y, v);
		rangeApply(2 * p + 1, m, r, x, y, v);
		pull(p);
	}
	void rangeApply(int l, int r, const Tag &v) { return rangeApply(1, 0, n, l, r, v); }
};

int main() {
	std::ios::sync_with_stdio(false);
	std::cin.tie(nullptr);
	
	int n, m;
	std::cin >> n >> m;
	
	std::vector<Info> a(n);
	for (Info &i : a) {
		std::cin >> i.x;
	}
	
	LazySegmentTree<Info, int64_t> seg(a);
	
	while (m--) {
		int op;
		std::cin >> op;
		
		if (op == 1) {
			int x, y, k;
			std::cin >> x >> y >> k;
			--x;
			seg.rangeApply(x, y, k);
		} else {
			int x, y;
			std::cin >> x >> y;
			--x;
			
			std::cout << seg.rangeQuery(x, y).x << '\n';
		}
	}
}
\end{lstlisting}