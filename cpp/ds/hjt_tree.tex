先引入一道题目:给定 $n$ 个整数构成的序列 $a$,将对于指定的闭区间 $[l, r]$ 查询其区间内的第 $k$ 小值。

主席树的主要思想就是:保存每次插入操作时的历史版本,以便查询区间第 $k$ 小。

我们把问题简化一下:每次求 $[1,r]$ 区间内的 $k$ 小值。  

怎么做呢?只需要找到插入 r 时的根节点版本,然后用普通权值线段树(有的叫键值线段树/值域线段树)做就行了。

所以……如果需要得到 $[l,r]$ 的统计信息,只需要用 $[1,r]$ 的信息减去 $[1,l - 1]$ 的信息就行了。

至此,该问题解决!

关于空间问题,我们分析一下:由于我们是动态开点的,所以一棵线段树只会出现 $2n-1$ 个结点。  
然后,有 $n$ 次修改,每次至多增加 $\lceil\log_2{n}\rceil+1$ 个结点。因此,最坏情况下 $n$ 次修改后的结点总数会达到 $2n-1+n(\lceil\log_2{n}\rceil+1)$。
此题的 $n \leq 10^5$,单次修改至多增加 $\lceil\log_2{10^5}\rceil+1 = 18$ 个结点,故 $n$ 次修改后的结点总数为 $2\times 10^5-1+18\times 10^5$,忽略掉 $-1$,大概就是 $20\times 10^5$。

最后给一个忠告:千万不要吝啬空间(大多数题目中空间限制都较为宽松,因此一般不用担心空间超限的问题)!大胆一点,直接上个 $2^5\times 10^5$,接近原空间的两倍(即 "n << 5")。

\begin{lstlisting}
#include <algorithm>
#include <cstdio>
#include <cstring>
using namespace std;
const int maxn = 1e5;  // 数据范围
int tot, n, m;
int sum[(maxn << 5) + 10], rt[maxn + 10], ls[(maxn << 5) + 10],
rs[(maxn << 5) + 10];
int a[maxn + 10], ind[maxn + 10], len;

inline int getid(const int &val) {  // 离散化
	return lower_bound(ind + 1, ind + len + 1, val) - ind;
}

int build(int l, int r) {  // 建树
	int root = ++tot;
	if (l == r) return root;
	int mid = l + r >> 1;
	ls[root] = build(l, mid);
	rs[root] = build(mid + 1, r);
	return root;  // 返回该子树的根节点
}

int update(int k, int l, int r, int root) {  // 插入操作
	int dir = ++tot;
	ls[dir] = ls[root], rs[dir] = rs[root], sum[dir] = sum[root] + 1;
	if (l == r) return dir;
	int mid = l + r >> 1;
	if (k <= mid)
	ls[dir] = update(k, l, mid, ls[dir]);
	else
	rs[dir] = update(k, mid + 1, r, rs[dir]);
	return dir;
}

int query(int u, int v, int l, int r, int k) {  // 查询操作
	int mid = l + r >> 1,
	x = sum[ls[v]] - sum[ls[u]];  // 通过区间减法得到左儿子中所存储的数值个数
	if (l == r) return l;
	if (k <= x)  // 若 k 小于等于 x ,则说明第 k 小的数字存储在在左儿子中
	return query(ls[u], ls[v], l, mid, k);
	else  // 否则说明在右儿子中
	return query(rs[u], rs[v], mid + 1, r, k - x);
}

inline void init() {
	scanf("%d%d", &n, &m);
	for (int i = 1; i <= n; ++i) scanf("%d", a + i);
	memcpy(ind, a, sizeof ind);
	sort(ind + 1, ind + n + 1);
	len = unique(ind + 1, ind + n + 1) - ind - 1;
	rt[0] = build(1, len);
	for (int i = 1; i <= n; ++i) rt[i] = update(getid(a[i]), 1, len, rt[i - 1]);
}

int l, r, k;

inline void work() {
	while (m--) {
		scanf("%d%d%d", &l, &r, &k);
		printf("%d\n", ind[query(rt[l - 1], rt[r], 1, len, k)]);  // 回答询问
	}
}

int main() {
	init();
	work();
	return 0;
}
\end{lstlisting}