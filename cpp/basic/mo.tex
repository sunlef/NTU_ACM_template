假设 $n=m$,那么对于序列上的区间询问问题,如果从 $[l,r]$ 的答案能够 $O(1)$ 扩展到 $[l-1,r],[l+1,r],[l,r+1],[l,r-1]$(即与 $[l,r]$ 相邻的区间)的答案,那么可以在 $O(n\sqrt{n})$ 的复杂度内求出所有询问的答案。

离线后排序,顺序处理每个询问,暴力从上一个区间的答案转移到下一个区间答案(一步一步移动即可)。

对于区间 $[l,r]$, 以 $l$ 所在块的编号为第一关键字,$r$ 为第二关键字从小到大排序。

\begin{lstlisting}
inline void move(int pos, int sign) {
	// update nowAns
}

void solve() {
	BLOCK_SIZE = int(ceil(pow(n, 0.5)));
	sort(querys, querys + m);
	for (int i = 0; i < m; ++i) {
		const query &q = querys[i];
		while (l > q.l) move(--l, 1);
		while (r < q.r) move(r++, 1);
		while (l < q.l) move(l++, -1);
		while (r > q.r) move(--r, -1);
		ans[q.id] = nowAns;
	}
}
\end{lstlisting}

例题[「国家集训队」小 Z 的袜子

题目大意:
有一个长度为 $n$ 的序列 $\{c_i\}$。现在给出 $m$ 个询问,每次给出两个数 $l,r$,从编号在 $l$ 到 $r$ 之间的数中随机选出两个不同的数,求两个数相等的概率。

对于区间 $[l,r]$,以 $l$ 所在块的编号为第一关键字,$r$ 为第二关键字从小到大排序。

然后从序列的第一个询问开始计算答案,第一个询问通过直接暴力算出,复杂度为 $O(n)$,后面的询问在前一个询问的基础上得到答案。

具体做法:

对于区间 $[i,i]$,由于区间只有一个元素,我们很容易就能知道答案。然后一步一步从当前区间(已知答案)向下一个区间靠近。

我们设 $col[i]$ 表示当前颜色 $i$ 出现了多少次,$ans$ 当前共有多少种可行的配对方案(有多少种可以选到一双颜色相同的袜子),表示然后每次移动的时候更新答案——设当前颜色为 $k$,如果是增长区间就是 $ans$ 加上 $C_{col[k]+1}^2-C_{col[k]}^2$,如果是缩短就是 $ans$ 减去 $C_{col[k]}^2-C_{col[k]-1}^2$。

而这个询问的答案就是 $\displaystyle \frac{ans}{C_{r-l+1}^2}$。

这里有个优化:$\displaystyle C_a^2=\frac{a (a-1)}{2}$。

所以 $\displaystyle C_{a+1}^2-C_a^2=\frac{(a+1) a}{2}-\frac{a (a-1)}{2}=\frac{a}{2}\cdot (a+1-a+1)=\frac{a}{2}\cdot 2=a$。

所以 $C_{col[k]+1}^2-C_{col[k]}^2=col[k]$。

算法总复杂度:$O(n\sqrt{n} )$

下面的代码中 `deno` 表示答案的分母 (denominator),`nume` 表示分子 (numerator),`sqn` 表示块的大小:$\sqrt{n}$,`arr` 是输入的数组,`node` 是存储询问的结构体,`tab` 是询问序列(排序后的),`col` 同上所述。

\begin{lstlisting}
#include <algorithm>
#include <cmath>
#include <cstdio>
using namespace std;
const int N = 50005;
int n, m, maxn;
int c[N];
long long sum;
int cnt[N];
long long ans1[N], ans2[N];

struct query {
	int l, r, id;
	
	bool operator<(const query &x) const {  // 重载<运算符
		if (l / maxn != x.l / maxn) return l < x.l;
		return (l / maxn) & 1 ? r < x.r : r > x.r;
	}
} a[N];

void add(int i) {
	sum += cnt[i];
	cnt[i]++;
}

void del(int i) {
	cnt[i]--;
	sum -= cnt[i];
}

long long gcd(long long a, long long b) { return b ? gcd(b, a % b) : a; }

int main() {
	scanf("%d%d", &n, &m);
	maxn = sqrt(n);
	for (int i = 1; i <= n; i++) scanf("%d", &c[i]);
	for (int i = 0; i < m; i++) scanf("%d%d", &a[i].l, &a[i].r), a[i].id = i;
	sort(a, a + m);
	for (int i = 0, l = 1, r = 0; i < m; i++) {  // 具体实现
		if (a[i].l == a[i].r) {
			ans1[a[i].id] = 0, ans2[a[i].id] = 1;
			continue;
		}
		while (l > a[i].l) add(c[--l]);
		while (r < a[i].r) add(c[++r]);
		while (l < a[i].l) del(c[l++]);
		while (r > a[i].r) del(c[r--]);
		ans1[a[i].id] = sum;
		ans2[a[i].id] = (long long)(r - l + 1) * (r - l) / 2;
	}
	for (int i = 0; i < m; i++) {
		if (ans1[i] != 0) {
			long long g = gcd(ans1[i], ans2[i]);
			ans1[i] /= g, ans2[i] /= g;
		} else
		ans2[i] = 1;
		printf("%lld/%lld\n", ans1[i], ans2[i]);
	}
	return 0;
}
\end{lstlisting}