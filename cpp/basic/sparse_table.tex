定义 $dp[i][k]$ 表示左断点为 $i$,区间长度为 $2^k$ 的区间最值。递推关系为:
$$dp[i][k] = merge\{dp[i][k - 1], dp[i + (1 << (k - 1))][k]\}$$

$O(nlogn)$预处理、$O(1)$ 查询。
\begin{lstlisting}
int N, M;
std::cin >> N >> M;

std::vector<int> a(N);
for (int &i : a) std::cin >> i;

std::vector<int> LOG2(N + 1);
LOG2[0] = -1;
for (int i = 1; i <= N; ++i) 
	LOG2[i] = LOG2[i / 2] + 1;

int dp[N][21];
for (int i = 0; i < N; ++i) 
	dp[i][0] = a[i];

int p = LOG2[N];
// 可倍增区间的最大次方:2^p <= N

for (int k = 1; k <= p; ++k)
	for (int i = 0; i + (1 << k) - 1 < N; ++i)
		dp[i][k] = std::max(dp[i][k - 1], dp[i + (1 << (k - 1))][k - 1]);

while (M--) {
	int l, r;
	//求 [l, r] 中最大值
	std::cin >> l >> r;
	--l;
	
	int len = r - l;
	int k = LOG2[len];
	
	std::cout << std::max(dp[l][k], dp[r - (1 << k)][k]) << '\n';
}
\end{lstlisting}

ST类:
\begin{lstlisting}
struct ST {
	size_t n;
	std::vector<std::vector<int>> info;
	
	int merge(const int &lhs, const int &rhs) {  // TODO: 修改merge
		return std::max(lhs, rhs);
	}
	
	ST(const std::vector<int> &init) : n(init.size()), info(init.size() + 1, std::vector<int>(std::__lg(n) + 1, INT_MIN)) {  //TODO : 修改初值
		for (size_t x = 0; x <= std::__lg(n); ++x)
		for (int i = 0; i + (1ull << x) <= n; ++i)
		if (x == 0)
		info[i][x] = init[i];
		else
		info[i][x] = merge(info[i][x - 1], info[i + (1 << (x - 1))][x - 1]);
	}
	
	int query(int l, int r) {  //[l, r)
		int len = r - l;
		assert(len > 0);
		return merge(info[l][std::__lg(len)], info[r - (1 << std::__lg(len))][std::__lg(len)]);
	}
};
\end{lstlisting}