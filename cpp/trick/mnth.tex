\begin{lstlisting}
#include <cmath>
#include <cstdio>
#include <cstdlib>
#include <ctime>

const int N = 10005;
int n, x[N], y[N], w[N];
double ansx, ansy, dis;

double Rand() { return (double)rand() / RAND_MAX; }

double calc(double xx, double yy) {
	double res = 0;
	for (int i = 1; i <= n; ++i) {
		double dx = x[i] - xx, dy = y[i] - yy;
		res += sqrt(dx * dx + dy * dy) * w[i];
	}
	if (res < dis) dis = res, ansx = xx, ansy = yy;
	return res;
}

void simulateAnneal() {
	double t = 100000;
	double nowx = ansx, nowy = ansy;
	while (t > 0.001) {
		double nxtx = nowx + t * (Rand() * 2 - 1);
		double nxty = nowy + t * (Rand() * 2 - 1);
		double delta = calc(nxtx, nxty) - calc(nowx, nowy);
		if (exp(-delta / t) > Rand()) nowx = nxtx, nowy = nxty;
		t *= 0.97;
	}
	for (int i = 1; i <= 1000; ++i) {
		double nxtx = ansx + t * (Rand() * 2 - 1);
		double nxty = ansy + t * (Rand() * 2 - 1);
		calc(nxtx, nxty);
	}
}

int main() {
	srand(0);  // 注意,在实际使用中,不应使用固定的随机种子。
	scanf("%d", &n);
	for (int i = 1; i <= n; ++i) {
		scanf("%d%d%d", &x[i], &y[i], &w[i]);
		ansx += x[i], ansy += y[i];
	}
	ansx /= n, ansy /= n, dis = calc(ansx, ansy);
	simulateAnneal();
	printf("%.3lf %.3lf\n", ansx, ansy);
	return 0;
}
\end{lstlisting}