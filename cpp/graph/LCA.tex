倍增:
\begin{lstlisting}
#include <bits/stdc++.h>

int main() {
	std::ios::sync_with_stdio(false);
	std::cin.tie(nullptr);
	
	int n, m, s;
	std::cin >> n >> m >> s;
	--s;
	
	std::vector G(n, std::vector<int>());
	
	for (int i = 1; i < n; ++i) {
		int u, v;
		std::cin >> u >> v;
		--u, --v;
		
		G[u].emplace_back(v);
		G[v].emplace_back(u);
	}
	
	std::vector fa(n, std::vector<int>(20, -1));
	//fa[u][i] 表示u的第2^i个父亲
	std::vector<int> depth(n);
	
	std::function<void(int, int)> 
	dfs = [&](int u, int f) {
		depth[u] = depth[f] + 1;
		fa[u][0] = f;
		
		for (int i = 1; (1 << i) <= depth[u]; ++i) 
		fa[u][i] = fa[fa[u][i - 1]][i - 1];
		
		for (int v : G[u])
		if (v != f) dfs(v, u);
	};
	
	auto lca = [&](int x, int y) {
		if (depth[x] < depth[y]) std::swap(x, y);
		
		for (int i = 19; i >= 0; --i)
		if (depth[x] - (1 << i) >= depth[y]) 
		x = fa[x][i];
		
		if (x == y) return x;
		
		for (int i = 19; i >= 0; --i)
		if (fa[x][i] != fa[y][i]) {
			x = fa[x][i];
			y = fa[y][i];
		}
		
		return fa[x][0];
	};
	
	dfs(s, s);
	
	while (m--) {
		int x, y;
		std::cin >> x >> y;
		--x, --y;
		
		std::cout << lca(x, y) + 1 << '\n';
	}
}
\end{lstlisting}

targan:
\begin{lstlisting}
int n, root, cnt;
int pre[maxn], ans[maxn];
vector<int> v[maxn], s[maxn], num[maxn];

int find(int x) { return pre[x] == x ? x : pre[x] = find(pre[x]); }

void dfs(int u) {
	pre[u] = u;
	for (int i = 0; i < v[u].size(); i++) {
		int to = v[u][i];
		dfs(to);
		pre[find(pre[to])] = find(pre[u]);
	}
	for (int i = 0; i < s[u].size(); i++) {
		int to = s[u][i];
		if (pre[to] != to)
		ans[num[u][i]] = find(pre[to]);
	}
}

/*
for (int i = 1; i <= q; i++) {
	scanf("%d%d", &x, &y);
	if (x == y) ans[i] = x;
	s[x].push_back(y);
	s[y].push_back(x);
	num[x].push_back(i);
	num[y].push_back(i);
}
dfs(root);
*/
\end{lstlisting}