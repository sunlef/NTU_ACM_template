给定一个包含 n 个点 m 条边的有向图,并给定每条边的容量,边的容量非负。

图中可能存在重边和自环。求从点 S 到点 T 的最大流。

输入格式
第一行包含四个整数 n,m,S,T。

接下来 m 行,每行三个整数 u,v,c,表示从点 u 到点 v 存在一条有向边,容量为 c。

点的编号从 1 到 n。

输出格式
输出点 S 到点 T 的最大流。

如果从点 S 无法到达点 T 则输出 0。
\begin{lstlisting}
#include<bits/stdc++.h>

using namespace std;

#define int long long
#define endl '\n'

const int N = 1e4 + 10, M = 2e5 + 10, inf = 0x3f3f3f3f;

int n, m, st, ed;
int h[N], e[M], w[M], ne[M], idx;
int d[N], cu[N];

void add(int a, int b, int c) {
	e[idx] = b, w[idx] = c, ne[idx] = h[a], h[a] = idx++;
	e[idx] = a, w[idx] = 0, ne[idx] = h[b], h[b] = idx++;
}

bool bfs() {
	memset(d, -1, sizeof d); d[st] = 0;
	queue<int> q; q.push(st);
	cu[st] = h[st];
	while (q.size()) {
		int ver = q.front(); q.pop();
		for (int i = h[ver]; ~i; i = ne[i]) {
			int to = e[i];
			if (d[to] == -1 && w[i]) {
				d[to] = d[ver] + 1;
				cu[to] = h[to];
				if (to == ed) return 1;
				q.push(to);
			}
		}
	}
	return 0;
}

int Find(int pre, int limit) {
	if (pre == ed) return limit;
	int flow = 0;
	for (int i = cu[pre]; ~i && flow < limit; i = ne[i]) {
		cu[pre] = i;
		int to = e[i];
		if (d[to] == d[pre] + 1 && w[i]) {
			int tmp = Find(to, min(w[i], limit - flow));
			if (!tmp) d[to] = -1;
			w[i] -= tmp, w[i ^ 1] += tmp, flow += tmp;
		}
	}
	return flow;
}

int dinic() {
	int maxf = 0, flow;
	while (bfs()) while (flow = Find(st, inf)) maxf += flow;
	return maxf;
}

void oper() {
	cin >> n >> m >> st >> ed;
	memset(h, -1, sizeof h); idx = 0;
	for (int i = 1; i <= m; i++) {
		int a, b, c; cin >> a >> b >> c;
		add(a, b, c);
	}
	cout << dinic() << endl;
}

signed main() {
	ios::sync_with_stdio(0), cin.tie(0), cout.tie(0);
	int t = 1; //cin >> t;
	while (t--) oper();
	return 0;
}
\end{lstlisting}

\begin{lstlisting}
/*
* Author: Simon,Ttmq36
* 复杂度: O(V^2·E)
*/
namespace Dinic{
	struct node{
		int v,w,next;
		node(){}
		node(int v,int w,int next):v(v),w(w),next(next){}
	}g[maxn<<1];
	int head[maxn],cur[maxn],cnt=0;
	int dep[maxn];
	void init(){
		memset(head,-1,sizeof(head));cnt=0;
	}
	void addedge(int u,int v,int w){
		g[cnt]=node(v,w,head[u]);
		head[u]=cnt++;
		g[cnt]=node(u,0,head[v]);
		head[v]=cnt++;
	}
	bool bfs(int s,int t){
		memset(dep,0,sizeof(dep));
		queue<int>q;q.push(s);
		dep[s]=1; cur[s]=head[s];
		while(!q.empty()){
			int u=q.front();q.pop();
			for(int i=head[u];~i;i=g[i].next){
				int &v=g[i].v,&w=g[i].w;
				if(!dep[v]&&w>0){
					dep[v] = dep[u] + 1;
					cur[v] = head[v];
					q.push(v);
				}
			}
		}
		if(!dep[t]) return 0;
		return 1;
	}
	int dfs(int s,int t,int f){
		if(s==t) return f;
		int flow=0,u=s;
		for(int &i=cur[u];~i;i=g[i].next){
			int &v=g[i].v,&w=g[i].w;
			if(dep[u]+1==dep[v]&&w>0){
				int d=dfs(v,t,min(f,w));
				if(d>0){
					flow+=d; f-=d;
					g[i].w-=d,g[i^1].w+=d;
					if(f<=0) break;
				}
			}
		}
		if(!flow) dep[u]=-INF;
		return flow;
	}
	int maxFlow(int s,int t){
		int ans=0;
		while(bfs(s,t)){
			ans+=dfs(s,t,INF);
		}
		return ans;
	}
}
\end{lstlisting}