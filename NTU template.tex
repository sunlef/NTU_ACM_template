\documentclass[twocolumn,a4]{article}
\usepackage{indentfirst}
\usepackage{xeCJK} % For Chinese characters
\usepackage{amsmath, amsthm}
\usepackage{listings,xcolor}
\usepackage{geometry} % 设置页边距
\usepackage{fontspec}
\usepackage{graphicx}
\usepackage{fancyhdr} % 自定义页眉页脚
\setsansfont{Fira Code} % 设置英文字体
\setmonofont[Mapping={}]{Fira Code} % 英文引号之类的正常显示,相当于设置英文字体
\geometry{left=1cm,right=1cm,top=2cm,bottom=0.5cm} % 页边距
\setlength{\columnsep}{30pt}
\newcommand{\addcpp}[1]{\input{cpp/#1}} 
%\setlength\columnseprule{0.4pt} % 分割线

%========================页眉、页脚、代码格式设置=======================%
% 页眉、页脚设置
\pagestyle{fancy}
% \lhead{CUMTB}
\lhead{\CJKfamily{hei} Nantong University XCPC Template}
\chead{}
% \rhead{Page \thepage}
\rhead{\CJKfamily{hei} 第 \thepage 页}
\lfoot{} 
\cfoot{}
\rfoot{}
\renewcommand{\headrulewidth}{0.4pt} 
\renewcommand{\footrulewidth}{0.4pt}

% 代码格式设置
\lstset{
    language    = c++,
    numbers     = left,
    numberstyle = \small,
    breaklines  = true,
    captionpos  = b,
    tabsize     = 4,
    frame       = shadowbox,
    columns     = fixed,
    commentstyle = \color[RGB]{0,128,0},
	keywordstyle = \color[RGB]{0,51,153},
    basicstyle   = \small \ttfamily,
    stringstyle  = \color[RGB]{148,0,209}\ttfamily,
	rulesepcolor = \color{red!20!green!20!blue!20},
    showstringspaces = false,                    
}
%========================页眉、页脚、代码格式设置=======================%

%===============================标题和目录==============================%
\title{\CJKfamily{hei} \bfseries 「集思广益一波」}
\author{CleanBlue}
\renewcommand{\today}{\number\year 年 \number\month 月 \number\day 日}

\begin{document} \small
\begin{titlepage}
\maketitle
\end{titlepage}

\newpage
\pagestyle{empty}
\renewcommand{\contentsname}{目录}
\tableofcontents
\newpage\clearpage
\newpage
\pagestyle{fancy}
\setcounter{page}{1}   %new page
%===============================标题和目录==============================%

%================================正文部分===============================%
\section{基本算法}
	\subsection{快速幂/整数类/取模}
		\addcpp{basic/Z.tex}
	\subsection{一二三维差分/前缀和}
		\addcpp{basic/difference.tex}
	\subsection{二分}
		\addcpp{basic/binary_search.tex}
	\subsection{三分}
		\addcpp{basic/ternary_search.tex}
	\subsection{ST/离线区间最值}
		\addcpp{basic/sparse_table.tex }
	\subsection{归并排序/逆序对}
		\addcpp{basic/merge_sort.tex}
	\subsection{分块}
		\addcpp{basic/block.tex}
	\subsection{莫队}
		\addcpp{basic/mo.tex}
		
\section{数据结构}
	\subsection{单调队列/滑动窗口}
		\addcpp{ds/SlidingWindow.tex}
	\subsection{单调栈}
		\addcpp{ds/mono-stack.tex}
	\subsection{并查集}
		\addcpp{ds/dsu.tex}
	\subsection{可撤销并查集}
		\addcpp{graph/undodsu.tex}
	\subsection{树状数组}
		\addcpp{ds/fenwick.tex}
	\subsection{字典树}
		\addcpp{ds/trie.tex}
	\subsection{线段树}
		\addcpp{ds/segment_tree.tex}
	\subsection{可持久化线段树}
		\addcpp{ds/hjt_tree.tex}
		
\section{字符串}
	\subsection{字符串哈希}
		\addcpp{string/string_hash.tex}
	\subsection{KMP}
		\addcpp{string/KMP.tex}
	\subsection{exKMP}
		\addcpp{string/exKMP.tex}
	\subsection{Manacher}
		\addcpp{string/Manacher.tex}

\section{图论}
	\subsection{LCA}
		\addcpp{graph/LCA.tex}
	\subsection{最小生成树}
		\addcpp{graph/mst.tex}
	\subsection{最短路}
		\addcpp{graph/zuiduanlu.tex}
	\subsection{欧拉回路}
		\addcpp{graph/oulahuilu.tex}
	\subsection{K短路}
		\addcpp{graph/kth.tex}
	\subsection{强连通分量}
		\addcpp{graph/qltfl.tex}
	\subsection{次小生成树}
		\addcpp{graph/SMST.tex}
	\subsection{最大流}
		\addcpp{graph/zuidaliu.tex}
	\subsection{启发式合并}
		\addcpp{ds/qifashihebing.tex}
		

\section{动态规划}
	\subsection{背包}
		\addcpp{dp/backpack.tex}
	\subsection{最长上升子序列}
		\addcpp{dp/LIS.tex}
	\subsection{最长公共子序列}
		\addcpp{dp/LCS.tex}
	\subsection{数位dp}
		\addcpp{dp/numdp.tex}
	\subsection{区间dp}
		\addcpp{dp/rangedp.tex}		
		
\section{搜索}
	\subsection{双向搜索}
		\addcpp{search/double_search.tex}
	\subsection{启发式搜索}
		\addcpp{search/heuristic.tex}

\section{数学}
	\subsection{exgcd}
		\addcpp{math/exgcd.tex}
	\subsection{欧拉函数}
		\addcpp{math/eular.tex}
	\subsection{逆元}
		\addcpp{math/inv.tex}
	\subsection{组合数}
		\addcpp{math/comb.tex}
	\subsection{博弈论}
		\addcpp{math/game.tex}
	\subsection{容斥}
		\addcpp{math/rongchi.tex}
	\subsection{素性检验}
		\addcpp{math/prime_check.tex}
	\subsection{线性筛}
		\addcpp{math/sieve.tex}
	\subsection{常用公式}
		\addcpp{math/forulma.tex}
		
\section{trick}
	\subsection{离散化}
		\addcpp{trick/dirtri.tex}
	\subsection{bitset}
		\addcpp{trick/bitset.tex}
	\subsection{位运算}
		\addcpp{trick/bit.tex}
	\subsection{内置函数}
		\addcpp{trick/builtin.tex}
	\subsection{模拟退火}
		\addcpp{trick/mnth.tex}
	\subsection{几何公式}
		\addcpp{trick/geo.tex}
%==============================正文部分==============================%
\end{document}